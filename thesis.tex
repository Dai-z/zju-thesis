% !Mode:: "TeX:UTF-8"
% !TEX builder = LATEXMK
% !TEX program = xelatex

% TODO disable draft mode and enable anon if necessary
\documentclass[master,twoside,nocpsupervisor]{style/zjuthesis}
% \documentclass[anon,master,twoside,nocpsupervisor]{style/zjuthesis}

% 插图路径设置,图片放在figures 文件夹下。一般来说论文的插图比较多,通常按章节存
% 放,因此可以在以下命令中在按章节添加存放图片的文件夹路径。如以下这个路径中 ./
% 代表当前main.tex所在的目录,就是一般所说的当前文件夹;figures 文件夹就是子文件
% 夹,存放正文及附录中要用到的所有的图片,在figures 文件夹中的子文件夹就是存放各
% 个章节图片的文件夹,一般命名与相应章节的名字相同,如intro 章节用到的图片全放在
% 了intro 这个子文件夹下。
% \graphicspath{%
%     {./figures/intro/}%
% }
% !Mode:: "TeX:UTF-8"
% !TEX root = ../thesis.tex

% lowercase symbols
\newcommand{\bzero}{\textbf{0}}
\newcommand{\bone}{\textbf{1}}
\newcommand{\ba}{\mathbf{a}}
\newcommand{\bb}{\mathbf{b}}
\newcommand{\bc}{\mathbf{c}}
\newcommand{\bd}{\mathbf{d}}
\newcommand{\be}{\mathbf{e}}
\newcommand{\bff}{\mathbf{f}}
\newcommand{\bg}{\mathbf{g}}
\newcommand{\bh}{\mathbf{h}}
\newcommand{\bi}{\mathbf{i}}
\newcommand{\bn}{\mathbf{n}}
\newcommand{\bmm}{\mathbf{m}}
\newcommand{\bo}{\mathbf{o}}
\newcommand{\bp}{\mathbf{p}}
\newcommand{\bq}{\mathbf{q}}
\newcommand{\br}{\mathbf{r}}
\newcommand{\bs}{\mathbf{s}}
\newcommand{\bts}{\tilde{\mathbf{s}}}
\newcommand{\bt}{\mathbf{t}}
\newcommand{\bu}{\mathbf{u}}
\newcommand{\btu}{\tilde{\mathbf{u}}}
\newcommand{\bk}{\mathbf{k}}
\newcommand{\bv}{\mathbf{v}}
\newcommand{\bw}{\mathbf{w}}
\newcommand{\bx}{\mathbf{x}}
\newcommand{\btx}{\tilde{\mathbf{x}}}
\newcommand{\by}{\mathbf{y}}
\newcommand{\bz}{\mathbf{z}}

% uppercase symbols
\newcommand{\bA}{\mathbf{A}}
\newcommand{\bB}{\mathbf{B}}
\newcommand{\bC}{\mathbf{C}}
\newcommand{\bD}{\mathbf{D}}
\newcommand{\bF}{\mathbf{F}}
\newcommand{\btF}{\tilde{\mathbf{F}}}
\newcommand{\bG}{\mathbf{G}}
\newcommand{\bH}{\mathbf{H}}
\newcommand{\bI}{\mathbf{I}}
\newcommand{\bJ}{\mathbf{J}}
\newcommand{\bK}{\mathbf{K}}
\newcommand{\bL}{\mathbf{L}}
\newcommand{\bM}{\mathbf{M}}
\newcommand{\bN}{\mathbf{N}}
\newcommand{\bP}{\mathbf{P}}
\newcommand{\btP}{\tilde{\mathbf{P}}}
\newcommand{\bQ}{\mathbf{Q}}
\newcommand{\bR}{\mathbf{R}}
\newcommand{\btS}{\tilde{\mathbf{S}}}
\newcommand{\bS}{\mathbf{S}}
\newcommand{\bT}{\mathbf{T}}
\newcommand{\btT}{\tilde{\mathbf{T}}}
\newcommand{\bU}{\mathbf{U}}
\newcommand{\bV}{\mathbf{V}}
\newcommand{\bW}{\mathbf{W}}
\newcommand{\bX}{\mathbf{X}}
\newcommand{\bY}{\mathbf{Y}}
\newcommand{\bZ}{\mathbf{Z}}

\newcommand{\1}{\bmath{1}}
\newcommand{\0}{\bmath{0}}
\newcommand{\bmsigma}{\bm{\sigma}}
\newcommand{\bmmu}{\bm{\mu}}
\newcommand{\bmepsilon}{\bm{\epsilon}}
\newcommand{\bmgamma}{\bm{\gamma}}
\newcommand{\bmbeta}{\bm{\beta}}

% sf symbols
\newcommand{\sfD}{\mathsf{D}}
\newcommand{\sfU}{\mathsf{U}}

% hat symbols
\newcommand{\bhf}{\hat{\mathbf{f}}}
\newcommand{\bhh}{\hat{\mathbf{h}}}
\newcommand{\bhp}{\hat{\mathbf{p}}}
\newcommand{\bhx}{\hat{\mathbf{x}}}
\newcommand{\bhA}{\hat{\mathbf{A}}}
\newcommand{\bhc}{\hat{\mathbf{c}}}
\newcommand{\bhd}{\hat{\mathbf{d}}}
\newcommand{\bhF}{\hat{\mathbf{F}}}
\newcommand{\bhG}{\hat{\mathbf{G}}}
\newcommand{\bhH}{\hat{\mathbf{H}}}
\newcommand{\bhJ}{\hat{\mathbf{J}}}
\newcommand{\bhK}{\hat{\mathbf{K}}}

% mathcal symbols
\newcommand{\mA}{\mathcal{A}}
\newcommand{\mC}{\mathcal{C}}
\newcommand{\mD}{\mathcal{D}}
\newcommand{\mE}{\mathcal{E}}
\newcommand{\mF}{\mathcal{F}}
\newcommand{\mI}{\mathcal{I}}
\newcommand{\mK}{\mathcal{K}}
\newcommand{\mM}{\mathcal{M}}
\newcommand{\mN}{\mathcal{N}}
\newcommand{\mL}{\mathcal{L}}
\newcommand{\mP}{\mathcal{P}}
\newcommand{\mR}{\mathcal{R}}
\newcommand{\mS}{\mathcal{S}}
\newcommand{\mT}{\mathcal{T}}
\newcommand{\mmt}{\mathcal{t}}
\newcommand{\mU}{\mathcal{U}}
\newcommand{\mW}{\mathcal{W}}
\newcommand{\mmx}{\mathcal{x}}
\newcommand{\mZ}{\mathcal{Z}}

% mathbb symbols
\newcommand{\mbR}{\mathbb{R}} %symbol for the Real numbers
\newcommand{\mbE}{\mathbb{E}} %symbol for the Expectations

% other symbols
\newcommand{\Avvo}{\mathcal{A}}
\newcommand{\Tvvo}{\mathcal{T}}
\newcommand{\balpha}{\boldsymbol{\alpha}}
\newcommand{\bdelta}{\boldsymbol{\delta}}
\newcommand{\bDelta}{\boldsymbol{\Delta}}
\newcommand{\blambda}{\boldsymbol{\lambda}}
\newcommand{\bLambda}{\boldsymbol{\Lambda}}
\newcommand{\bmu}{\boldsymbol{\mu}}
\newcommand{\bgamma}{\boldsymbol{\gamma}}
\newcommand{\bGamma}{\boldsymbol{\Gamma}}
\newcommand{\bSigma}{\boldsymbol{\Sigma}}
\newcommand{\btheta}{\boldsymbol{\theta}}
\newcommand{\bTheta}{\boldsymbol{\Theta}}
\newcommand{\brho}{\boldsymbol{\rho}}
\newcommand{\bphi}{\boldsymbol{\phi}}
\newcommand{\bPhi}{\boldsymbol{\Phi}}
\newcommand{\bpsi}{\boldsymbol{\psi}}
\newcommand{\bPsi}{\boldsymbol{\Psi}}
\newcommand{\bxi}{\boldsymbol{\xi}}
\newcommand{\bUpsilon}{\boldsymbol{\Upsilon}}
\newcommand{\bomega}{\boldsymbol{\omega}}
\newcommand{\bOmega}{\boldsymbol{\Omega}}

\newcommand{\mo}{\boldsymbol{o}}

\newcommand{\berror}{\boldsymbol{\varepsilon}}

\newcommand{\st}{{\;t}}
\newcommand{\sk}{{\;k}}
\newcommand{\tms}{\hspace{-1mm}\times{\hspace{-1mm}}}
\newcommand{\hh}{\hspace{-1mm}}

\newcommand{\argmin}{\operatornamewithlimits{arg\,min}}
\newcommand{\argmax}{\operatornamewithlimits{arg\,max}}

% references
\newcommand{\figref}[1]{图\,\ref{#1}\,}
\newcommand{\tabref}[1]{表\,\ref{#1}\,}
\newcommand{\chapref}[1]{第\ref{#1}章}
\newcommand{\secref}[1]{第\,\ref{#1}\,节}
\newcommand{\ssubref}[1]{第\,\ref{#1}\,小节}
\newcommand{\algref}[1]{算法\,\ref{#1}\,}
\newcommand{\eqnref}[1]{式\,\ref{#1}\,}
\newcommand{\etal}{et al.}

% !Mode:: "TeX:UTF-8"
% !TEX root = ../thesis.tex

\newcommand{\imageplaceholder}{
  \begin{figure}[ht]
    \centering
    \includegraphics[width=0.95\linewidth]{example-image-duck}
    \caption{Image placeholder.}
  \end{figure}
}

\newcommand{\tinyimageplaceholder}{
  \begin{figure}[ht]
    \centering
    \includegraphics[width=0.50\linewidth]{example-image-duck}
    \caption{Image placeholder.}
  \end{figure}
}


% 论文中文标题
\title{研究生毕业论文总结报告\LaTeX{}模板}
% 论文英文标题
\englishtitle{\LaTeX{} Template for Thesis}
% 作者,就是你的名字
\author{郝仁}
% 分类号
\classification{TM863}
% 单位代码
\serialnumber{10335}
% 密级
\secretlevel{公开}
% 学号
\studentnumber{54321}
% 指导教师
\supervisor{渡鸦12345}
% 合作导师,如果没有合作导师,就在\documentclass选项栏中加上"nocpsupervisor"。
\cpsupervisor{无}
% 专业名称
\major{时空管理局}
% 研究方向
\research{世界和平}
% 所在学院
\institute{希灵帝国}
% 提交日期
\submitdate{\today}

% 中文题名页
\reviewerA{}
\reviewerB{}
\reviewerC{}
\reviewerD{}
\reviewerE{}
\chairperson{}
\commissionerA{}
\commissionerB{}
\commissionerC{}
\commissionerD{}
\commissionerE{}
\defencedate{\today}

% 英文题名页
\enreviewerA{}
\enreviewerB{}
\enreviewerC{}
\enreviewerD{}
\enreviewerE{}
\enchairperson{}
\encommissionerA{}
\encommissionerB{}
\encommissionerC{}
\encommissionerD{}
\encommissionerE{}
\eendefencedate{\today}

\begin{document}

\maketitle
% \ZJUmakecover
% \ZJUmakeCNtitlepage
% \ZJUmakeENtitlepage

\frontmatter
% !Mode:: "TeX:UTF-8"
% !TEX root = ../thesis.tex

\ifanon
\else
\chapter{致\texorpdfstring{\ZJUspace}{}谢}
岁月如梭,转眼间,为时两年半的摸鱼研究生生涯即将结束。
在此期间,我认识了许多关心、爱护、帮助我的人们,他们激励着我在求学路上奋勇向前。
浙江大学学习风气优良、科研氛围严谨、校园生活充实。
正是在此种环境与人们的影响下,我才能顺利进行科研与生活。
值此毕业论文完成之际,谨向所有关心、爱护、帮助我的人们表示最诚挚的感谢与最美好的祝愿。

首先,需要感谢我的导师老咸鱼教授。

其次,要感谢实验室的各位鱼类同学。

最后,感谢父母在我求学生涯中一如既往地理解、支持与鼓励我。
这份亲情是我求学最大的动力,激励我能更加坚定地走向前方、拥抱梦想。

\vspace{2cm}
\hfill
\begin{minipage}{14em}
\begin{center}
于浙江大学\quad 2020年1月10日\\
咸鱼
\end{center}
\end{minipage}
\fi

% !Mode:: "TeX:UTF-8"
% !TEX root = ../thesis.tex

% 定义中文摘要和关键字
\begin{cabstract}
请注意,以下内容是参考自\textbf{薛瑞尼}的清华大学论文模板,主要是为了填内容方便。

论文的摘要是对论文研究内容和成果的高度概括。摘要应对论文所研究的问题及其研究目
的进行描述,对研究方法和过程进行简单介绍,对研究成果和所得结论进行概括。摘要应
具有独立性和自明性,其内容应包含与论文全文同等量的主要信息。使读者即使不阅读全
文,通过摘要就能了解论文的总体内容和主要成果。

论文摘要的书写应力求精确、简明。切忌写成对论文书写内容进行提要的形式,尤其要避
免“第 1 章……;第 2 章……;……”这种或类似的陈述方式。

本文介绍浙江大学论文模板的使用方法。本模板符合学校的硕士、博士论文格式要求。
写这个模板的主要原因是想深入学习一下\LaTeX,还有可以自己毕业的时候用。

本文的创新点主要有:
\begin{itemize}
    \item 用例子来解释模板的使用方法;
    \item 用废话来填充无关紧要的部分;
    \item 一边学习摸索一边编写新代码。
\end{itemize}

关键词是为了文献标引工作、用以表示全文主要内容信息的单词或术语。关键词不超过 5
个,每个关键词中间用分号分隔。(模板作者注:关键词分隔符不用考虑,模板会自动处
理。英文关键词同理。)
\end{cabstract}

\ckeywords{\TeX, \LaTeX, CJK, 模板, 毕业论文}

% !Mode:: "TeX:UTF-8"
% !TEX root = ../thesis.tex

% 定义英文摘要和关键字

\begin{eabstract}
An abstract of a dissertation is a summary and extraction of research work
and contributions. Included in an abstract should be description of research
topic and research objective, brief introduction to methodology and research
process, and summarization of conclusion and contributions of the
research. An abstract should be characterized by independence and clarity and
carry identical information with the dissertation. It should be such that the
general idea and major contributions of the dissertation are conveyed without
reading the dissertation.

An abstract should be concise and to the point. It is a misunderstanding to
make an abstract an outline of the dissertation and words ``the first
chapter'', ``the second chapter'' and the like should be avoided in the
abstract.

Key words are terms used in a dissertation for indexing, reflecting core
information of the dissertation. An abstract may contain a maximum of 5 key
words, with semi-colons used in between to separate one another.
\end{eabstract}

\ekeywords{\TeX, \LaTeX, CJK, template, thesis}


% 正文目录:
\tableofcontents
% 插图目录:
% \listoffigures
% 表格目录:
% \listoftables
% \include{contents/denotation}

\mainmatter

% !Mode:: "TeX:UTF-8"
% !TEX root = ../thesis.tex

\chapter{绪论}
\label{cha:introduction}

\imageplaceholder

\cite{Kingma2014AdamAM}

\section{研究背景}
\label{sec:background}
% section 研究背景 (end)

\section{国内外研究现状}
\label{sec:related_works}
% section 国内外研究现状 (end)

\section{本文研究内容}
\label{sec:thesis_target}
% section 本文研究内容 (end)

\section{本文结构安排}
\label{sec:thesis_structure}
% section 本文结构安排 (end)
% chapter 绪论 (end)


\backmatter

\bibliography{thesis}
% \nocite{*} % to show the entire references, annotate it if need.

% \appendix
% \include{thesis/appendixA}
% \include{thesis/appendixB}
\end{document}
